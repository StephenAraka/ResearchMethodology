\documentclass{article}                    % article class
 
\begin{document}                           % Begin document text
 
\section{{\textbf{Who Is Elon Musk?}}}                % Print a section heading
Elon Reeve Musk (born June 28, 1971) is a South African-born American entrepreneur and businessman who founded X.com in 1999 (which later became PayPal), SpaceX in 2002 and Tesla Motors in 2003. Musk became a multimillionaire in his late 20s when he sold his start-up company, Zip2, to a division of Compaq Computers. Musk made headlines in May 2012, when SpaceX launched a rocket that would send the first commercial vehicle to the International Space Station. He bolstered his portfolio with the purchase of SolarCity in 2016, and cemented his standing as a leader of industry by taking on an advisory role in the early days of President Donald Trump's administration.
\section{{\textbf{Elon Musk’s Net Worth}}}
\textit{As of December 2017, Elon Musk’s net worth is 20.2 billion dollars, according to Forbes. He earned his first billion with the sale of PayPal in 2002; his company SpaceX is valued at more than 20 billion dollars.} and 
\textbf{Education}.

{\small At age 17, in 1989, Elon Musk moved to Canada to attend Queen’s University and avoid mandatory service in the South African military. He left in 1992 to study business and physics at the University of Pennsylvania. He graduated with an undergraduate degree in economics and stayed for a second bachelor’s degree in physics.
.}After leaving Penn, Elon Musk headed to Stanford University in California to pursue a PhD in energy physics. However, his move was timed perfectly with the Internet boom, and he dropped out of Stanford after just two days to become a part of it, launching his first company, Zip2 Corporation. An online city guide, Zip2 was soon providing content for the new websites of both The New York Times and the Chicago Tribune. In 1999, a division of Compaq Computer Corporation bought Zip2 for 307 million dollars in cash and 34 million dollars in stock options. 
\textbf{PayPal}
\textit{In 1999, Musk co-founded X.com, an online financial services/payments company. An X.com acquisition the following year led to the creation of PayPal as it is known today, and in October 2002, PayPal was acquired by eBay for 1.5 billion dollars in stock. Before the sale, Musk owned 11 percent of PayPal stock.}
  
\subsection{Founder of SpaceX}                        % Print a subsection heading
Musk founded his third company, Space Exploration Technologies Corporation, or SpaceX, in 2002 with the intention of building spacecraft for commercial space travel. By 2008, SpaceX was well established, and NASA awarded the company the contract to handle cargo transport for the International Space Station—with plans for astronaut transport in the future—in a move to replace NASA’s own space shuttle missions.

Falcon 9 Rockets
On May 22, 2012, Musk and SpaceX made history when the company launched its Falcon 9 rocket into space with an unmanned capsule. The vehicle was sent to the International Space Station with 1,000 pounds of supplies for the astronauts stationed there, marking the first time a private company had sent a spacecraft to the International Space Station. Of the launch, Musk was quoted as saying, "I feel very lucky. ... For us, it's like winning the Super Bowl."

In December 2013, a Falcon 9 successfully carried a satellite to geosynchronous transfer orbit, a distance at which the satellite would lock into an orbital path that matched the Earth's rotation. In February 2015, SpaceX launched another Falcon 9 fitted with the Deep Space Climate Observatory (DSCOVR) satellite, aiming to observe the extreme emissions from the sun that affect power grids and communications systems on Earth.

In March 2017, SpaceX saw the successful test flight and landing of a Falcon 9 rocket made from reusable parts, a development that opened the door for more affordable space travel. A setback came in November 2017, when an explosion occurred during a test of the company's new Block 5 Merlin engine. SpaceX reported that no one was hurt, and that the issue would not hamper its planned rollout of a future generation of Falcon 9 rockets.

The company enjoyed another milestone moment in February 2018 with the successful test launch of the powerful Falcon Heavy rocket. Armed with additional Falcon 9 boosters, the Falcon Heavy was designed to carry immense payloads into orbit and potentially serve as a vessel for deep space missions. For the test launch, the Falcon Heavy was given a payload of Musk's cherry-red Tesla Roadster, equipped with cameras to "provide some epic views" for the vehicle's planned orbit around the sun. 
\subsection{Elon Musk’s Wives}                        % Print a subsection heading 
Elon Musk has been married twice. He wed Justine Wilson in 2000. After a contentious divorce, Musk met actress Talulah Riley, and the couple married in 2010. They split in 2012 but married each other again in 2013. Their relationship ultimately ended in divorce in 2016. Musk has also been involved in an on-again, off-again relationship with actress Amber Heard.
\section{Kids}
Elon Musk has five sons with ex-wife Justine Wilson. In 2002, his first son with died at 10 weeks old from sudden infant death syndrome (SIDS). Musk and Wilson had five additional sons together: twins and triplets.
\newpage

\section{{\textbf{Birthday and Nationality}}}                % Print a section heading
Elon Musk was born on June 28, 1971 in Pretoria, South Africa. He became a U.S. citizen in 2002. 
\section{{\textbf{Family and Early Life}}}
\textit{Son of a Canadian mother and a South African father, Elon Musk spent his early childhood with his brother Kimbal and sister Tosca in South Africa. At 10, around the time his parents divorced, the introverted Elon developed an interest in computers. He taught himself how to program, and when he was 12 he made his first software sale of a game he created called Blastar.} and 
\textbf{Other Inventions and Innovations}.

{\small Outside of his roles at SpaceX and Tesla, Musk has continually attempted to make his innovative ideas a reality.

Hyperloop
In August 2013, Elon Musk released a concept for a new form of transportation called the "Hyperloop," an invention that would foster commuting between major cities while severely cutting travel time. Ideally resistant to weather and powered by renewable energy, the Hyperloop would propel riders in pods through a network of low-pressure tubes at speeds reaching more than 700 mph. Musk noted that the Hyperloop could take from seven to 10 years to be built and ready for use.

Although he introduced the Hyperloop with claims that it would be safer than a plane or train, with an estimated cost of 6 billion dollars — approximately one-tenth of the cost for the rail system planned by the state of California — Musk's concept has drawn skepticism. Nevertheless, the entrepreneur has sought to encourage the development of this idea. After he announced a competition for teams to submit their designs for a Hyperloop pod prototype, the first Hyperloop Pod Competition was held at the SpaceX facility in January 2017.

AI
Elon Musk has pursued an interest in Artificial Intelligence, becoming co-chair of the nonprofit OpenAI. The research company launched in late 2015 with the stated mission of advancing digital intelligence to benefit humanity. In 2017, it was also revealed that Musk was backing a venture called Neuralink, which intends to create devices to be implanted in the human brain and help people merge with software.

Boring Company
In yet another innovation, in January 2017 Elon Musk suddenly decided he was going find a way to reduce traffic by devoting resources to boring and building tunnels. He launched his venture, named "The Boring Company," with a test dig on the SpaceX property in Los Angeles. In late October, Musk posted the first photo of his company's progress to his Instagram page. He said the 500-foot tunnel, which would generally run parallel to Interstate 405, would reach a length of two miles in approximately four months.

The entrepreneur also reportedly found a market for the Boring Company's flamethrowers; after announcing they were going on sale for 500 dollars apiece in late January 2018, he claimed to have sold 10,000 of them within a day.

High-Speed Train
In late November 2017, after Chicago Mayor Rahm Emanuel asked for proposals to build and operate a high-speed rail line that would transport passengers from O'Hare Airport to downtown Chicago in 20 minutes or less, Musk tweeted that he was all-in on the competition with his Boring Company. He said that the concept of the Chicago loop would be different from his Hyperloop, its relatively short route not requiring the need for drawing a vacuum to eliminate air friction. 
 
\textbf{Research Type} 
\textit{QUALITATIVE} research
\end{document}                             % The required last line
